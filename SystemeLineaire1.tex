\documentclass[12pt, a4paper, openany]{report}

\def\VersionRapport{1.0}

\usepackage[utf8]{inputenc} % un package
\usepackage[T1]{fontenc}      % un second package
\usepackage[francais]{babel}  % un troisième package
\usepackage{layout}
\usepackage[top=2.7cm, bottom=2.5cm, left=3.5cm, right=3cm]{geometry}
\usepackage{setspace}

\frenchbsetup{StandardLists=true} % à inclure si on utilise \usepackage[french]{babel}
%\usepackage{enumitem}
\usepackage[shortlabels]{enumitem}
\usepackage{amssymb}

\usepackage{color}
\usepackage{listings}
\definecolor{dkgreen}{rgb}{0,0.6,0}
\definecolor{gray}{rgb}{0.5,0.5,0.5}
\definecolor{mauve}{rgb}{0.58,0,0.82}

\lstset{frame=tb,
  language=Java,
  aboveskip=3mm,
  belowskip=3mm,
  showstringspaces=false,
  columns=flexible,
  basicstyle={\small\ttfamily},
  numbers=none,
  numberstyle=\tiny\color{gray},
  keywordstyle=\color{blue},
  commentstyle=\color{dkgreen},
  stringstyle=\color{mauve},
  breaklines=true,
  breakatwhitespace=true,
  tabsize=3
}

\usepackage{multirow} % pour les tableaux
\usepackage[table]{xcolor} % pour les tableaux

\usepackage{verbatim}
\usepackage{moreverb}
\usepackage{url}
\usepackage{pst-all}
\usepackage{eso-pic,graphicx}
\usepackage{caption} 
\usepackage[colorlinks=true,urlcolor=blue,linkcolor=red]{hyperref}
\usepackage{array}
\usepackage[toc,page]{appendix}
\usepackage[off]{auto-pst-pdf}
\usepackage{hyperref} % pour le sommaire table des matières
\AddThinSpaceBeforeFootnotes % à insérer si on utilise \usepackage[french]{babel}
\FrenchFootnotes % à insérer si on utilise \usepackage[french]{babel}
\usepackage{fancyhdr}
\pagestyle{headings}
\usepackage{pifont}  %pour les puces
\usepackage{amsmath} %pour les puces

\usepackage{verbatim} % pour le code en annexe 

%%%%%%%colones 
\newcolumntype{R}[1]{>{\raggedleft\arraybackslash }b{#1}}
\newcolumntype{L}[1]{>{\raggedright\arraybackslash }b{#1}}
\newcolumntype{C}[1]{>{\centering\arraybackslash }b{#1}}
%%%%%%% 

\renewcommand{\appendixpagename}{Annexes}
\renewcommand{\appendixtocname}{Annexes}

\title{Theme: Compte Rendu Système Linéaire à Temps Continu 1}
\author{REBOUT \bsc{Mehenna}}
\author{BOUYOUCEF \bsc{Farid}}
\date{2018-2019}



%new
\newcommand{\HRule}{\rule{\linewidth}{0.5mm}}


\begin{document}

%\selectlanguage{francais}
\pagenumbering{arabic} 

\makeatletter
  \begin{titlepage}
  

  \begin{sffamily}
   \begin{center}

    % Upper part of the page. The '~' is needed because \\
    % only works if a paragraph has started.
    \includegraphics[scale=0.5]{Logo_UT3.jpg}~\\[1.5cm]

    \textsc{\LARGE Master 1 EEA ISTR/RODECO  }\\[2cm]

    \textsc{\Large Compte Rendu  Système Linéaire à Temps Continu 1}\\[1.5cm]

    % Title
    \HRule \\[0.4cm] % saut de ligne
    { \huge \bfseries TP 1 Pendule\\[0.4cm] }

    \HRule \\[1cm]   % sous de ligne 
    \includegraphics[scale=0.1]{logomaster.jpg}
    \\[1cm]

    % Author and supervisor
    \begin{minipage}{0.4\textwidth}
      \begin{flushleft} \large
         \textsc{\emph {Réalisés par:} \\REBOUT Mehenna}\\
         \textsc{BOUYOUCEF Farid}   
          \newline
          Promotion 2018-2019 \\
      \end{flushleft}
    \end{minipage}
    \begin{minipage}{0.4\textwidth}
      \begin{flushright} \large
        \emph{Tuteur :}  \textsc{M DUROLA}\\
        \emph{Responsable de la Formation:} \textsc{M GOUAISBAUT}
      \end{flushright}
    \end{minipage}

    \vfill

    % Bottom of the page
    {\large Octobre 2018}

  \end{center}
  \end{sffamily}      
          
  \end{titlepage}
  
\makeatother




   
%*********************** somaire **************
\renewcommand{\contentsname}{Sommaire}
\tableofcontents
%*********************** listes des figures **************
\listoffigures
%*********************** listes des tableaux **************
\listoftables



%*********************** INTRODUCTION **************
\chapter*{Introduction}
\addcontentsline{toc}{chapter}{Introduction}

\paragraph{Présentation du pendule inversé :} La Figure 1 donne une représentation schématique du pendule. Le bras est relié au moteur, qui entraîne la rotation du bras et donc du pendule. Il a une longueur L{r} et un moment d'inertie J{r} . Nous noterons son angle $\theta$ . Le pendule, quant à lui, est axé au bout du bras. Il a une longueur L{p}. Son moment d'inertie au centre de masse est noté J{p}. On note l'angle que fait l'axe du pendule avec l'axe verticale a (en radian).

\begin{center}
\includegraphics[scale=0.5]{pondule.png}
\captionof{figure}{\textit{Composant du pendule}}
\label{fig1} 
\end{center}

\paragraph{Précision du lieu :} Ce TP a eu lieux a l’université Paul SABATIER dans la salle TP I3.   

\paragraph{La date :} Ce sujet nous a été attribué par M DUROLA le mois d'octobre  et on va le rendre le 19/11/2018 à M DUROLA.

\paragraph{But de la manipulation :}  Dans cette manipulation on se propose de réaliser un asservissement permettant la régulation du pendule inversé représentée par la Figure 1.1 en position haute comme le montre la Figure 1.2. On utilisera pour cela les techniques d'espace d'état à temps continu. Cette manipulation est constituée de quatre grandes étapes : modélisation, analyse, régulation et implémentation. Les objectifs du TP sont les suivants :
\begin{itemize}
\item Savoir établir un modèle linéarisé autour d'un point d'équilibre.
\item Savoir établir les propriétés structurelle d'un système linéaire.
\item Savoir utiliser Matlab et sa toolbox simulink.
\item Savoir synthétiser une commande par retour d'état.
\item Savoir faire la simulation de modèles linéaires et non linéaires.
\item Savoir utiliser les différentes commandes MATLAB pour étudier un système.

\end{itemize}

    
    %******************* MODÉLISATION **********
 \chapter{Modélisation}
 
 La Figure 1.1 donne une représentation schématique du pendule. Le bras est relié au moteur, qui entraîne la rotation du bras et donc du pendule. Il a une longueur $L_{r}$ et un moment d'inertie $J_{r}$. Nous noterons son angle $\theta$.\\
 Le pendule, quant à lui, est lixé au bout du bras. Il a une longueur $L_{p}$ . Son moment d'inertie au
centre de masse est noté $J_{p}$. On note l'angle que fait l'axe du pendule avec l'axe verticale $\alpha$ (en radian).\\
 Tous les calculs de la modélisation se feront en prenant pour origine, la position haute du pendule. pour les descriptions et les valeurs numériques des constantes se reporter à la Table (num du table....) \\
 Les équations du mouvement, déterminées à l'aide de la mécanique Lagrangienne, sont données par les
deux équations différentielles suivantes : \\[1cm]\\

 %\big(a+b \big) \\ 
 %\bigg(a+b+c \bigg) \\
 %\Big(a+b+c+d \Big)\\
 %\Bigg(a+bjhbnjkhnkjn \Bigg)\\

  $\bigg(m_{p}L_{r}^{2} + \frac{1}{4}m_{p}L_{p}^{2} - \frac{1}{4}m_{p}L_{p}^{2}cos(\alpha^{2}) + J_{r} \bigg)$

   


\section{Étude succinte du modèle non linéaire  }

                                                 


%%%%%%%%%%%%%%%%%%%%%%%%% Tableau%%%%%%%%%%%%%%%%%%%ù


\begin{center}
\begin{tabular}{|R{2cm}|C{8cm}|L{2.5cm}|}
\hline \rowcolor{lightgray} Symbole & Description &  Valeur  \\
\hline  \begin{center} $B_{p}$\end{center}  &  \begin{center}Coefficient de frottement visqueux du pendule\end{center}  & \begin{center}$0.0024 kg/m^{2}$ \end{center}  \\
\hline  \begin{center} $B_{r}$\end{center} & \begin{center}Coefficient de frottement visqueux du bras\end{center} & \begin{center}$0.0024 kg/m^{2}$ \end{center}\\
\hline  \begin{center} $\eta_{g}$\end{center}  &  \begin{center}Rendement du réducteur\end{center}  & \begin{center}$0.9$ \end{center}  \\
\hline  \begin{center} $\eta_{m}$\end{center}  &  \begin{center}Rendement du moteur\end{center}  & \begin{center}$0.69$ \end{center}  \\
\hline  \begin{center} $K_{g}$\end{center}  &  \begin{center}Rapport de vitesse\end{center}  & \begin{center}$70$ \end{center}  \\

\hline 
\end{tabular}
\captionof{table}{Mehenna Rebout ihh a farid}
\end{center}


%%%%%%%%%%%%%%%%%%%%%%%%%%%%%%%%%%%%%%%




 %\begin{enumerate}[(a)]
  % \item Le système sera stable en boucle fermée
   %\item Une erreur de position nulle
   %\item Une erreur de vitesse limitée à 1 i.e. lorsque l'entrée de consigne est une rampe de pente 1
   %\item Une convergence vers la consigne en moins de 3 seconde, sans oscillations, ni dépassement.
   %ù\item Un rejet de la perturbation de fuite.
   %ù\item Un rejet du bruit de mesure au-delà de 100 Hz d’au moins de 60 dB par.
 %\end{enumerate}
	
%*********************** Problématique **************
\chapter{Analyse temporelle et structurelle du modèle linéarisé }
 %\chaptermark{Cmd proportionnelle intégrales} (pour avoir ce titre en petit 
 
 \section{Schéma bloc de l'asservissement}  %**** Question 1 ***** 
   % captures d’écrans 
   
   
   \begin{center}
   $G(p)=\frac {h_{1}(p)}{q_{e}(p)}=\frac{16p+20}{p^{3}+7p^{2}+12.75p+6.75}$   
   \end{center}
   
   On considère le premier correcteur de la forme : \\
   \begin{center}
   $U(p)=K\frac {1+\tau_{i}p}{\tau_{i}p}\xi(p)$ 
   \\[2cm]  
   \end{center}  
   
  
  
   
  
 \section{Débit de fuite est constant}  %**** Question 2  *****
 
 
 On utilisant les lois de la Mécanique Des Fluides, la pression est proportionnelle au débit volumique, est on constatant que l'eau est incompressible donc le volume ne varie pas par rapport au temps, d’où le débit volumique soit constant, l'hypothèse est valable. 
 
 
 \section{Le Diagramme asymptotique de $K(p)$ }  %**** Question 3  *****
 
 On a  $G(p)=\frac {U(p)}{\xi(p)}=K\frac {1+\tau_{i}p}{\tau_{i}p}$\\
 
 est on obtient le diagramme suivant:\\
 
 
 \section{Les spécifications} %***** Question 4  *******

 D’après la première commande on a un correcteur Proportionnel Intégrale, on obtient alors un déphasage de -$\pi$ en base fréquence et aussi un gain infini, et selon le cahier de charge pour avoir une convergence vers la consigne en moins de 3 seconde, il nous faut alors une bande passante élevée, ainsi un effet intégrateur afin d'avoir une erreur de position nulle. (correcteur PI $\Rightarrow$ erreur de position nulle)
 
 
 \section{Les contraintes sur le Gabarit 1 ainsi sur le Gabarit 2}  %**** Question 5  *****
 
 Dans ces deux Gabarits, on va déterminer ses contraintes sur la fonction de sensibilité permettant de satisfaire les spécifications b) et c) de notre cahier de charge.\\[0.1cm]\\
  %%***************** Gabarit 1 *********
 \begin{itemize}[label=\ding{108},font=\small\color{black}]
\item Pour une erreur de position nulle on a:\\
$\xi_{p}= \underset{p\longrightarrow 0}{lim}\hspace{2mm} p\hspace{2mm}\xi(p)$ \\
        =$\underset{p\longrightarrow 0}{lim}\hspace{2mm} p\hspace{2mm} S(p)\hspace{2mm} r(p)$ \hspace{2cm} tel que: \hspace{0.5cm}$r(p)=\frac {1}{p}$ \\
        =$\underset{p\longrightarrow 0}{lim}\hspace{2mm} p\hspace{2mm} S(p)\hspace{2mm} \frac {1}{p} $\\ 
        =S(0)=0\\
        \\[1mm]\\
        Si on choisit $w_{1}(p)=\frac {1}{p} $ \hspace{1cm} tel que :\hspace{2mm}|$w_{1}(p).S(p)|<1$\\
        $\Rightarrow$ |$S(p)|<p$ \hspace{3mm} et on remplaçant dans la limite on obtient : \\ 
        $\xi_{p}=\underset{p\longrightarrow 0}{lim}\hspace{2mm} p\hspace{2mm} p \hspace{2mm} \frac {1}{p} $\\
        %**\\[0.2cm]
        $=\underset{p\longrightarrow 0}{lim}\hspace{2mm} p\hspace{2mm}$\\
        =\hspace{2mm}0  \\
        \\[0.5cm]\\
       $\Rightarrow$  \hspace{2mm} $|S(jw)|_{H}$<$jw$  \hspace{2mm} tel que : $|S(jw)|_{H}$ est la valeur maximum que peut atteindre le module de $S(p)$ \\
       donc : $w_{1}(p)=\frac {1}{p} $ \hspace{2mm} est un gabarit nécessaire pour une erreur de position nulle.\\
  \end{itemize}
    %*********** Gabarit 2  *******
  \begin{itemize}[label=\ding{108},font=\small\color{black}]
  \item Pour une erreur de vitesse limitée à 1 lorsque l'entrée de consigne est une rampe de pente 1 on a :\\ 
 $\xi_{v}= \underset{p\longrightarrow 0}{lim}\hspace{2mm} p\hspace{2mm}\xi(p)$ \\
    \hspace{2cm}=$\underset{p\longrightarrow 0}{lim}\hspace{2mm} p\hspace{2mm} S(p)\hspace{2mm} r(p)$ \hspace{2cm} tel que: \hspace{0.5cm}$r(p)=\frac {1}{p^{2}}$ \\
     \\[1mm]\\
        Si on choisit $w_{2}(p)=\frac {1}{p} $ \hspace{1cm} tel que :\hspace{2mm}|$w_{2}(p).S(p)|<1$\\
        $\Rightarrow$ |$S(p)|<p$ \hspace{3mm} et on remplaçant dans la limite on obtient alors : \\ 
        $\xi_{v}=\underset{p\longrightarrow 0}{lim}\hspace{2mm} p\hspace{2mm} p \hspace{2mm} \frac {1}{p^{2}} $\\
        $=\underset{p\longrightarrow 0}{lim}\hspace{2mm} \frac {p^{2}}{p^{2}}\hspace{2mm}$\\
        =\hspace{2mm}1 $\leq$1 \\
        \\[5mm]\\
        $\Rightarrow$  \hspace{2mm} $|S(jw)|_{H}$<$jw$  \hspace{2mm} tel que : $|S(jw)|_{H}$ est la valeur maximum que peut atteindre le module de $S(p)$ \\
       donc : $w_{2}(p)=\frac {1}{p} $ \hspace{2mm} est un gabarit nécessaire pour une erreur de vitesse limitée à 1 lorsque l'entrée de consigne est une rampe de pente 1.\\
  \end{itemize} 

\section{La spécification sur la vitesse de convergence ainsi le Gabarit 3 pour la sensibilité}  %**** Question 6 et 7 *****

D'après notre cahier de charge, sur le système on veut avoir une convergence vers la consigne en mois de 3 seconde, et pour que le système soit plus rapide selon la bande passante on pose alors comme une contrainte:\\
 $|T(jw)|_{db}-|T(0)|_{db} \hspace{0.2cm} > \hspace{0.2cm} 3_{db}$\\ 
 \\[0.07cm]\\
 et pour avoir une condition suffisante, alors d'après la contrainte supposée ci-dessus on a :\\
 \\[1mm]\\
 20.log($|T(jw)|_{db}-|T(0)|_{db}$) \hspace{0.2cm} > \hspace{0.2cm} 20.logS($3_{db}$)\\
 
$\Rightarrow\hspace{0.2cm} \frac {|T(jw)|}{|T(0)|} \hspace{0.2cm} > \hspace{0.2cm} 10^{3/20} \hspace{0.2cm}=1/\sqrt{2} \hspace{0.5cm} avec : \hspace{0.2cm} |T(0)|=1$ \\

$\Rightarrow\hspace{0.2cm} |T(jw)| \hspace{0.2cm} > \hspace{0.2cm} 1/\sqrt{2}$ \\

on choisit alors :$\hspace{0.3cm} |S(jw)|\hspace{0.2cm}\leq \hspace{0.2cm} 1-(1/\sqrt{2}).......(1)$ \\
 
et on a aussi: $\hspace{0.3cm} |T(jw)+S(jw)|\hspace{0.2cm}= \hspace{0.2cm} 1$\\

alors: $\hspace{0.3cm} |T(jw)|+|S(jw)|\hspace{0.2cm}\geq \hspace{0.2cm} 1........(2)$\\  

de (1) on a : $1/\sqrt{2}\leq 1-|S(jw)|........(3)$ \\

et aussi de (2) $|T(jw)|\geq 1-|S(jw)|.......(4)$ \\

alors de (3) et de (4) on obtient: \\
 
 $\hspace{0.3cm} 1/\sqrt{2}\hspace{0.2cm} \leq 1-|S(jw)| \hspace{0.2cm}\leq |T(jw)|$ \\
 
 $\Rightarrow \hspace{0.3cm} |T(jw)| \hspace{0.2cm} \geq 1/\sqrt{2}$ \\
 
donc la condition est suffisante, et cela on va calculer le module en dB de la fonction de sensibilité selon ce choit : \\

on a alors : $|S(jw)|\hspace{0.2cm}\leq 1-(1/\sqrt{2}) \hspace{0.3cm} , \forall w\in[0,w_{c}] $ \\ 

 $20.log(1-(1/\sqrt{2})\hspace{0.2cm} \simeq \hspace{0.2cm} -10.67$  \\
 
 $\Rightarrow |S(jw)|_{db}\hspace{0.2cm} \leq \hspace{0.2cm} -10.67 \hspace{0.3cm} , \forall w\in[0,w_{c}] $  \\
 
 donc : le Gabarit 3 est nécessaire pour la fonction de sensibilité quelque soit $w\in[0,w_{c}]. $   
 
 
\section{La spécification selon les oscillations}  %**** Question 8 ***** 

selon notre cahier de charge, il est indiqué que des oscillations ne sont pas souhaitables,\\

alors la marge de module soit: $\hspace{0.2cm} M_{m}>1$\\

$M_{m}\hspace{0.2cm}=\hspace{0.2cm}\frac {1}{|S(jw)|_{H}}\hspace{0.2cm}\geq 1$ \\

$\Rightarrow |S(jw)|\hspace{0.2cm} < \hspace{0.2cm} 1 $ \\

On choisit :  $w_{3}(jw)=1 \hspace{0.3cm} tel que: \hspace{0.2cm} |w_{3}(jw).S(jw)|\hspace{0.2cm}<\hspace{0.2cm} 1$ \\
  
   $\Rightarrow \hspace{0.2cm} |S(jw)|_{H}\hspace{0.2cm}< \hspace{0.2cm} 1$ \\

 $\Rightarrow \hspace{0.2cm} |S(jw)|_{dB} \hspace{0.2cm}< \hspace{0.2cm} 0$ \\
 
mais on a pas satisfait l'intégrale de Bode, et on remarquant que notre système est de 2 ème ordre,\\

et puisque on veut pas avoir des oscillations sur notre système, donc le coefficient d'amortissement $\psi$ soit: $\psi \hspace{0.2cm}\geq \hspace{0.2cm} 0.7$ \\

et on choisit: $\psi \hspace{0.2cm} = \hspace{0.2cm} 0.79$ \\

et on calcule le dépassement D pour $\psi \hspace{0.2cm} = \hspace{0.2cm} 0.79$ :\\


$ D=100\exp(\frac{-\psi\pi}{\sqrt{1-\psi^{2}}})  \hspace{0.2cm}=\hspace{0.2cm} 1.745$ \\

$\Rightarrow \hspace{0.2cm} 20log(1.109)\hspace{0.2cm} = \hspace{0.2cm} 4.83_{dB} $\\



\section{La spécification sur le rejet du bruit de mesure (Gabarit 4)}

on souhait un rejet du bruit de mesure au dela de 100 Hz d'au moins 60 dB par décade, \\

$ T|(jw)|\hspace{0.2cm}<\hspace{0.2cm} |\frac{1}{w_{5}}|  \hspace{0.2cm}$ , $\forall w\hspace{0.2cm}>2*\pi*100$ \\

on a: $|w_{5}|\hspace{0.2cm}=\hspace{0.2cm}  60dB $\\
$\Rightarrow \hspace{0.2cm} |T(jw)|\hspace{0.2cm}<\hspace{0.2cm}-60 dB $\\

$\Rightarrow |T(jw)| \hspace{0.2cm}<\hspace{0.2cm} 10^{\frac{-60}{20}} \hspace{0.2cm}=0.001$\\

$\Rightarrow |T(jw)| \hspace{0.2cm}<\hspace{0.2cm} 0.001 $\\

et on sait que : $|T(jw)|+|S(jw)| \hspace{0.2cm} \geq 1$ \\

  $|S(jw)|\hspace{0.2cm} \geq 1\hspace{0.2cm}-|T(jw)| \hspace{0.2cm} = 1\hspace{0.2cm}-\hspace{0.2cm}0.999 $ \\
  
$\Rightarrow |S(jw)|\hspace{0.2cm}\geq\hspace{0.2cm}0.999 ......(1)$ \\

et on a aussi: $\hspace{0.2cm} \hspace{0.2cm} S+T=1 $ 

$ |S|=|1-T|\hspace{0.2cm} \leq 1 +|T|\hspace{0.2cm}=1+0.001 $\\

$\Rightarrow |S(jw)|\hspace{0.2cm}\leq 1.001 ......(2) $ \\

donc : d'après (1) et (2) on a obtenue un intervalle sur le maximum et le minimum d'amplitude dans le canal de bande passante sur la sensibilité. \\

et on a tracer le diagramme de Bode qui englobe tous les Gabarit que on avez utiliser sur la fonction de sensibilité.   



\chapter{Réalisation d'une commande Loop-shaping }
 %\addcontentsline{toc}{chapter}{Réalisation d'une commande Loop-shaping}
 \chaptermark{Réalisation d'une commande Loop-shaping}
 
 Le rapport à compléter pour la partie d'une commande Loop-shaping, ainsi quelques détails pour le chapitre précédent (chapitre 2)   


% *********************** Conclusion *****************
\chapter*{Conclusion}
\addcontentsline{toc}{chapter}{Conclusion}

Ce TP nous a permet d'utiliser les différentes commandes de robustesse et a appris de définir un correcteur par étapes, cela en vérifiant chaque contrainte pour déterminer un meilleur correcteur.\\

\paragraph{Remarque :}
 L’importance de ce travail nous a motivé a travailler à fond, mais la gestion de notre temps sur ce travail nous a pas laissé terminer le travail complètement.


 %\chapter*{Annexe}
% \addcontentsline{toc}{chapter}{Introduction}

% CODE MATLAB


%\begin{verbatim}



 

% \end{verbatim}






%Bibliographie 
%\bibliographystyle{alpha}
%\bibliography{biblio}



\end{document}





